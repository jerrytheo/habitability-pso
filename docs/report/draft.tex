\documentclass{article}

\usepackage[utf8]{inputenc}
\usepackage{mathtools}
\usepackage{algorithm}
\usepackage{algpseudocode}


% Information for header.
\title{Optimizing the Cobb Doughlas Habitability Production function using
Particle Swarm Optimization}
\author{Abhijit J.\ Theophilus}
\date{\today}

% Useful for initial drafts.
\newenvironment{pointers}{%
  \noindent Should include,
  \begin{itemize}
    \setlength{\itemsep}{-1pt}}{%
\end{itemize}}

% Useful replacements.
\newcommand{\pso}{Particle Swarm Optimization}

% For algorithms
\floatname{algorithm}{Procedure}
\renewcommand{\algorithmicrequire}{\textbf{Input:}}
\renewcommand{\algorithmicensure}{\textbf{Output:}}



\begin{document}
\maketitle

\section{Introduction}
% Note to self, improve the flow of the report.
% Find citations.
Under most paradigms, optimization of a continuous function requires the calculation of a gradient. This might not
always be feasible for non-polynomial functions in high-dimensional search spaces. This is further complicated when the
optimization must be performed under constraints. \pso\ is a metaheuristic for continuous function optimization that does
away with the need for a gradient. It employs a large number of particles that traverse the search space converging
toward a global best solution encountered by at least one of the particles.

\pso\ is a distributed method that requires simple mathematical operators and few lines of code, making it a lucrative
solution where computation resources are at a premium. Its implementation is highly parallelizable. It scales with the
dimensionality of the search space. Constrained optimization can be incorporated within the standard algorithm through
the use of a penalty function.

% Pull out citations from the Applications paper.
\pso\ has been successfully adapted to a wide range of design optimization problems, e.g., networks and VLSI design.
\pso\ has also found applications in machine learning under clustering, feature detection and classification.  As a
modelling paradigm, it has been used for constructing customer satisfaction models, MIDI music models, friction models,
etc.

In this paper, \pso\ has been applied to estimate the Cobb Douglas Habitability Score (CDHS) of an exoplanet. The score
is estimated by maximizing the Cobb-Douglas Habitability Production Function (CD-HPF). The function considers four
parameters, radius, mass, escape velocity and surface temperature, taken from the Exoplanets Catalog hosted by the
Planetary Habitability Laboratory, UPR Arecibo. Since the Cobb-Douglas function is concave under certain constraints,
the problem can be reduced to a convex optimization over the negative of the production function.


\section{Cobb-Douglas Habitability Score}
% Add the citations to Saha's papers.

\subsection{Cobb-Douglas Production Function}
The general form of the Cobb-Douglas production function (CD-PF) is given by,
\begin{equation}
  Y = k\cdot{(x_1)}^\alpha\cdot{(x_2)}^\beta,
\end{equation}
where $k$ is a constant that may be set according to requirement, $x_1$ and $x_2$ are the input parameters to the
function, $Y$ is the total output production value, and $\alpha$ and $\beta$ are called the elasticity coefficients.

The CDHS is estimated by maximizing the CD-PF when the returns to scale are either decreasing or constant. This is
marked by two constraints on the elasticity coefficients,
\begin{itemize}
\setlength\itemsep{0pt}
\item When $\alpha+\beta < 1$, the CD-PF observes decreasing returns to scale (DRS).
\item When $\alpha+\beta = 1$, the CD-PF observes constant returns to scale (CRS).
\end{itemize}
Further, if $0 < \alpha, \beta < 1$, the CD-PF is concave, and therefore has a maxima. The CD-PF can be adapted as a
Habitability Production Function by setting the input parameters to the measurable parameters of an exoplanet and using
these to estimate the elasticity coefficients.

\subsection{Cobb-Douglas Habitability Score Estimation}
Two types of Habitability Scores are calculated for each planet, the interior CDHS (CDHS\textsubscript{i}) and the
surface CDHS (CDHS\textsubscript{s}), given by the following equations,
\begin{subequations}
  \begin{alignat}{4}
    Y_1 &=\ {CDHS}_i\ &=\ &D^\alpha\ & \cdot\ &R^\beta\ &,\\
    Y_2 &=\ {CDHS}_s\ &=\ &{V_e}^\gamma\ & \cdot\ &{T_s}^\delta\ &,
  \end{alignat}
\end{subequations}
where, $D$, $R$, $V_e$ and $T_s$ are density, radius, escape velocity and surface temperature respectively. The final
CDHS score is convex combination of the interior and surface CDHS values as given by,
\begin{equation}
  Y = w'\cdot Y_1 + w''\cdot Y_2,
\end{equation}

\begin{algorithm}
  \begin{algorithmic}
    \Require A dataset of exoplanets with density, radius, escape velocity and surface temperature in Earth Units.
    \Ensure The $CDHS$ value for each exoplanet in the input dataset.
    
    \State $CDHS \gets \{\}$
    \ForAll{exoplanets $e$}
      \State $D \gets$ $e.density$
      \State $R \gets$ $e.radius$
      \State $V_e \gets$ $e.escape\_velocity$
      \State $T_s \gets$ $e.surface\_temperature$
      \\ 
      \State $Y_1 \gets \max\ D^\alpha\cdot R^\beta,\quad \text{subject to:}\quad \alpha+\beta<1;\quad 0<\alpha,\beta<1$
      \State $Y_2 \gets \max\ {V_e}^\gamma\cdot {T_s}^\delta,\quad \text{subject to:}\quad \gamma+\delta<1;\quad
      0<\gamma,\delta<1$
      \State $Y.{DRS} \gets w'\cdot Y_1\ +\ w''\cdot Y_2$
      \\
      \State $Y_1 \gets \max\ D^\alpha\cdot R^\beta,\quad \text{subject to:}\quad \alpha+\beta=1;\quad 0<\alpha,\beta<1$
      \State $Y_2 \gets \max\ {V_e}^\gamma\cdot {T_s}^\delta,\quad \text{subject to:}\quad \gamma+\delta=1;\quad
      0<\gamma,\delta<1$
      \State $Y.{CRS} \gets w'\cdot Y_1\ +\ w''\cdot Y_2$
      \\
      \State append $Y$ to $CDHS$
    \EndFor
  \end{algorithmic}
  \caption{Estimating CDHS values.}
  \label{estCDHS}
\end{algorithm}

\section{Particle Swarm Optimization}
\begin{pointers}
\item Equations. Modification for max.
\item Constrained optimization modifications.
\item Algorithm for PSO\@.
\end{pointers}


\section{Experiment}
\begin{pointers}
\item Discussing the data set.
\item Parameters for the constrained optimization.
\item Implementation method.
\item Ensuring convergence.
\end{pointers}


\section{Results}
\begin{pointers}
\item The values and proximity to earth's habitability score.
\item Values that do not converge. Or were hard to converge.
\item Speed of convergence graphs.
\item Graph1: Iterations to convergence vs. Number of particles.
\item Graph2: Distribution of number of iterations to convergence.
\item Graph3: Iterations to convergence vs. Constraint parameters.
\end{pointers}


\section{Conclusions}
\begin{pointers}
\item Why is the speed so important?
\item Parallelizable.
\end{pointers}


\end{document}
