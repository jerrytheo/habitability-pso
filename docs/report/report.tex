\documentclass[10pt]{article}

\usepackage[
  citestyle=numeric-comp,
  sorting=none]{biblatex}
\usepackage[margin=1.2in]{geometry}
\usepackage[utf8]{inputenc}
\usepackage[english]{babel}

\usepackage{algorithm}
\usepackage{algpseudocode}

\usepackage[title]{appendix}
\usepackage{booktabs}
\usepackage{csquotes}
\usepackage{enumitem}

\usepackage{graphicx}
\usepackage{subcaption}

\usepackage{mathtools}
\usepackage{bm}
\usepackage{siunitx}

\graphicspath{{figs/}}
\sisetup{output-exponent-marker=\ensuremath{\mathrm{e}}}


% Information for header.
\title{Estimating Planetary Habitability via Particle Swarm Optimization of CES Production Functions.}
\author{Abhijit J.\ Theophilus}
\date{\today}


% Useful replacements.
\newcommand{\pso}{Particle Swarm Optimization}
\newcommand{\tsup}{\textsuperscript}


% For algorithms
\floatname{algorithm}{Procedure}
\renewcommand{\algorithmicrequire}{\textbf{Input:}}
\renewcommand{\algorithmicensure}{\textbf{Output:}}


% For math
\DeclareMathOperator*{\argmin}{arg\,min}
\DeclareMathOperator*{\argmax}{arg\,max}
\DeclareMathOperator*{\sgn}{sgn}
\DeclareMathOperator*{\pbest}{\mathit{pbest}}
\DeclareMathOperator*{\gbest}{\mathit{gbest}}
\DeclareMathOperator*{\oldbest}{\mathit{oldbest}}
\DeclareMathOperator*{\lbest}{\mathit{lbest}}


% Bibliography
\addbibresource{bibliography.bib}


\begin{document}
\maketitle


\section{Introduction}\label{sec:intro}

The search for extra-terrestrial life~\cite{Seth1,Biosig} and potentially habitable extrasolar
planets~\cite{Kepler,Presto} has been an international venture since Frank Drake's attempt with Project Ozma in the
mid-20th century~\cite{Ozma}. \textcite{Exo1} confirmed the first exoplanet in 1991. This marked the start of a trend
that has lasted 25 years and yielded over 3,700 confirmed exoplanets. There have been attempts to assess the
habitability of these planets and to assign a score based on their similarity to Earth. Two such habitability scores are
the Cobb-Douglas Habitability Score (CDHS)~\cite{Bora,Saha} and the Constant Elasticity Earth Similarity Approach
(CEESA) score. Estimating these scores involves maximizing a production function while observing a set of constraints on
the input variables.

Under most paradigms, maximizing a continuous function requires calculating a gradient. This may not always be feasible
for non-polynomial functions in high-dimensional search spaces. Further, subjecting the input variables to constraints,
as needed by CDHS and CEESA, are not always straightforward to represent within the model. This paper details an
approach to Constrained Optimization (CO) using the swarm intelligence metaheuristic. \pso\ (PSO) is a method for
optimizing a continuous function that does away with the need for calculating the gradient. It employs a large number of
randomly initialized particles that traverse the search space, eventually converging at a global best solution
encountered by at least one particle~\cite{PSO1,PSO2}.

\pso\ is a distributed method that requires simple mathematical operators and short segments of code, making it a
lucrative solution where computational resources are at a premium. Its implementation is highly parallelizable. It
scales with the dimensionality of the search space. The standard PSO algorithm does not deal with constraints but,
through variations in initializing and updating particles, constraints are straightforward to represent and adhere to,
as seen in Section~\ref{subsec:copso}. \textcite{Poli1,Poli2} carried out extensive surveys on the applications of PSO,
reporting uses in Communication Networks, Machine Learning, Design, Combinatorial Optimization and Modeling, among
others.

This paper demonstrates the applicability of \pso\ in estimating the CDHS and CEESA scores of an exoplanet by maximizing
their respective production functions (discussed in Sections~\ref{subsec:cdhs} and~\ref{subsec:ceesa}). CDHS considers
the planet's Radius, Mass, Escape Velocity and Surface Temperature, while CEESA includes a fifth parameter, the Orbital
Eccentricity of the planet. The Exoplanets Catalog hosted by the Planetary Habitability Laboratory, UPR Arecibo records
these parameters for each exoplanet in Earth Units~\cite{PHL}. Section~\ref{sec:results} reports the performance of PSO
and discusses the distribution of the habitability scores of the exoplanets.


\section{Habitability Scores}\label{sec:habscore}

\subsection{Cobb-Douglas Habitability Score}\label{subsec:cdhs}
Estimating the Cobb-Douglas Habitability Score (CDHS)~\cite{Bora} requires estimating an interior score \linebreak
($\mathit{CDHS}_i$) and a surface score ($\mathit{CDHS}_s$) by maximizing the following production functions,
\begin{subequations}
  \begin{alignat}{4}
    Y_i\ &=\ {CDHS}_i\ &=&\ R^\alpha.&D^\beta\,,\label{eq:cdhsi}\\
    Y_s\ &=\ {CDHS}_s\ &=&\ {V_e}^\gamma.&{T_s}^\delta\,,\label{eq:cdhss}
  \end{alignat}
\end{subequations}
where, $R$, $D$, $V_e$ and $T_s$ are density, radius, escape velocity and surface temperature respectively. $\alpha$,
$\beta$, $\gamma$ and $\delta$ are the elasticity coefficients subject to $0 < \alpha,\beta,\gamma,\delta < 1$.
Equations~\ref{eq:cdhsi} and~\ref{eq:cdhss} are convex under either Constant Returns to Scale (CRS), when
$\alpha+\beta=1$ and $\gamma+\delta=1$, or Decreasing Returns to Scale (DRS), when $\alpha+\beta<1$ and
$\gamma+\delta<1$. The final CDHS is the convex combination of the interior and surface scores as given by,
\begin{equation}
  Y\ =\ w_i.Y_i + w_s.Y_s\,.
\end{equation}

\subsection{Constant Elasticity Earth Similarity Approach}\label{subsec:ceesa}
% Cite what?
The Constant Elasticity Earth Similarity Approach (CEESA) uses the following production function to estimate the
habitability score of an exoplanet,
\begin{equation}\label{eq:ceesa}
  Y = {(r.R^\rho+d.D^\rho+t.{T_s}^\rho+v.{V_e}^\rho+e.E^\rho)}^{\frac{\eta}{\rho}}\,,
\end{equation}
where, $E$ is the fifth parameter denoting Orbital Eccentricity. The value of $\rho$ lies within $0<\rho\leq 1$. The
coefficients $r$, $d$, $t$, $v$ and $e$ lie in $(0,1)$ and sum to \num{1}, $r+d+t+v+e = 1$. The value of $\eta$ is
constrained by the scale of production used, $0 < \eta < 1$ under DRS and $\eta=1$ under CRS.


\section{Particle Swarm Optimization}\label{sec:pso}

Particle Swarm Optimization (PSO)~\cite{PSO1} is a biologically inspired metaheuristic for finding the global minima of
a function. Traditionally designed for unconstrained inputs, it works by iteratively converging a population of randomly
initialized solutions, called particles, toward a globally optimal solution. Each particle in the population keeps track
of its current position and the best solution it has encountered, called $\pbest$. Each particle also has an associated
velocity used to traverse the search space. The swarm keeps track of the overall best solution, called $\gbest$. Each
iteration of the swarm updates the velocity of the particle towards its $\pbest$ and the $\gbest$ values.

\subsection{PSO for Unconstrained Optimization}\label{subsec:uopso}
Let $f(x)$ be the function to be minimized, where $x$ is a $d$-dimensional vector. $f(x)$ is also called the fitness
function. Algorithm~\ref{alg:unop} outlines the approach to minimizing $f(x)$ using PSO.\@ A set of particles are randomly
initialized with a position and a velocity, where $l$ and $u$ are the lower and upper boundaries of the search space.
The position of the particle corresponds to its associated solution. The algorithm initializes each particle's $\pbest$
to its initial position. The $\pbest$ position that corresponds to the minimum fitness is selected to be the $\gbest$
position of the swarm. \textcite{PSO2} discussed the use of inertial weights to regulate velocity to balance the global
and local search. Upper and lower bounds limit velocity within $\pm v_\mathit{max}$.

On each iteration, the algorithm updates the velocity and position of each particle. For each particle, it picks two
random numbers $u_g, u_p$ from a uniform distribution, $U(0,1)$ and updates the particle velocity as indicated in
line~\ref{algline:vup}. Here, $\omega$ is the inertial weight and $\lambda_g,\lambda_p$ are the global and particle
learning rates. If the new position of the particle corresponds to a better fit than its $\pbest$, the algorithm updates
$\pbest$ to the new position. Once the algorithm has updated all particles, it updates $\gbest$ to the new overall best
position. A suitable termination criteria for the swarm, under convex optimization, is to terminate when $\gbest$ has
not changed by the end of an iteration.

\begin{algorithm}[t]
  \begin{algorithmic}[1]
    \Require{$f(x)$, the function to minimize.}
    \Ensure{global minimum of $f(x)$.}
    \For{each particle $i\gets 1,n$}
    \State{$p_i \sim {U(l,u)}^d$}
    \State{$v_i \sim {U(-|u-l|, |u-l|)}^d$}
      \State{${\pbest}_i \gets p_i$}
    \EndFor\
    \State{$\gbest \gets \smashoperator{\argmin\limits_{{\pbest}_i,\,i=1\dots n}} f({\pbest}_i)$}
    \Repeat\
      \State{$\oldbest \gets \gbest$}
      \For{each particle $i \gets 1\dots n$}
        \State{$u_p, u_g \sim U(0,1)$}
        \State{$v_i \gets \omega.v_i + \lambda_g.u_g.({\gbest-p_i}) + \lambda_p.u_p.({{\pbest}_i-p_i})$}\label{algline:vup}
        \State{$v_i \gets \sgn(v_i).\max\{|v_\mathit{max}|, |v_i|\}$}
        \State{$p_i \gets p_i + v_i$}
        \If{$f(p_i) < f({\pbest}_i)$}
          \State{${\pbest}_i \gets p_i$}
        \EndIf\
      \EndFor\
      \State{$\gbest \gets \smashoperator{\argmin\limits_{{\pbest}_i,\,i=1\dots n}} f({\pbest}_i)$}
    \Until{$|\oldbest - \gbest| < \mathit{threshold}$}
    \State{\textbf{return} {$f(\gbest)$}}
  \end{algorithmic}
  \caption{Algorithm for PSO.}\label{alg:unop}
\end{algorithm}


\subsection{PSO with Leaders for Constrained Optimization}\label{subsec:copso}
Although PSO has eliminated the need to estimate the gradient of a function, as seen in Section~\ref{subsec:uopso}, it
still is not suitable for constrained optimization. The standard PSO algorithm does not ensure that the initial
solutions are feasible, and neither does it guarantee that the individual solutions will converge to a feasible global
solution. Solving the initialization problem is straightforward, resample each random solution from the uniform
distribution until every initial solution is feasible. To solve the convergence problem each particle uses another
particle's $\pbest$ value, called $\lbest$, instead of its own to update its velocity. Algorithm~\ref{alg:cop} describes
this process.

On each iteration, for each particle, the algorithm first picks two random numbers $u_g,u_p$. It then selects a $\pbest$
value from all particles in the swarm that is closest to the position of the particle being updated as its $\lbest$. The
$\lbest$ value substitutes ${\pbest}_i$ in the velocity update equation. While updating $\pbest$ for the particle, the
algorithm checks if the current fit is better than $\pbest$, and performs the update if the current position satisfies
all constraints. The algorithm updates $\gbest$ as before.

\begin{algorithm}[t]
  \begin{algorithmic}[1]
    \Require{$f(x)$, the function to minimize.}
    \Ensure{global minimum of $f(x)$.}
    \For{each particle $i\gets 1,n$}
      \Repeat\
      \State{$p_i \sim {U(l,u)}^d$}
      \Until{$p_i$ satisfies all constraints}
      \State{$v_i \sim {U(-|u-l|, |u-l|)}^d$}
      \State{${\pbest}_i \gets p_i$}
    \EndFor\
    \State{$\gbest \gets \smashoperator{\argmin\limits_{{\pbest}_i,\,i=1\dots n}} f({\pbest}_i)$}
    \Repeat\
      \State{$\oldbest \gets \gbest$}
      \For{each particle $i \gets 1\dots n$}
        \State{$u_p, u_g \sim U(0,1)$}
        \State{$\lbest \gets \smashoperator{\argmin\limits_{{\pbest}_j,\,j=1\dots n}} {\|{\pbest}_j -
            p_i\|}^2$}\label{algline:lbest}
        \State{$v_i \gets \omega.v_i + \lambda_g.u_g.({\gbest-p_i}) + \lambda_p.u_p.({\lbest-p_i})$}\label{algline:cvup}
        \State{$p_i \gets p_i + v_i$}
        \If{$f(p_i) < f({\pbest}_i)$ \textbf{and} $p_i$ satisfies all constraints}
          \State{${\pbest}_i \gets p_i$}
        \EndIf\
      \EndFor\
      \State{$\gbest \gets \smashoperator{\argmin\limits_{{\pbest}_i,\,i=1\dots n}} f({\pbest}_i)$}
    \Until{$|\oldbest - \gbest| < \mathit{threshold}$}
    \State{\textbf{return} {$f(\gbest)$}}
  \end{algorithmic}
  \caption{Algorithm for CO by PSO.}\label{alg:cop}
\end{algorithm}


\section{Representing the Problem}\label{sec:rep}

A Constrained Optimization problem can represented as,
\begin{equation*}
  \begin{aligned}
    & \underset{x}{\text{minimize}}
    & & f(x) \\
    & \text{subject to}
    & & g_k(x) \leq 0,&\; k = 1\dots q\,,\\
    &&& h_l(x) = 0,&\quad l = 1\dots r\,.
  \end{aligned}
\end{equation*}

\textcite{Cop} describe a way to represent non-strict inequality constraints when optimizing using a particle swarm.
Strict inequalities and equality constraints need to be converted to non-strict inequalities before being represented in
the problem. Introducing an error threshold $\epsilon$ converts strict inequalities of the form ${g_k}^{\prime}(x) < 0$
to non-strict inequalities of the form $g_k(x) = {g_k}^{\prime}(x) + \epsilon \leq 0$. A tolerance $\tau$ is used to
transform equality constraints to a pair of inequalities,
\begin{equation*}
  \begin{aligned}
    g_{(q+l)}(x) &= h_l(x) - \tau \leq 0,&&\quad l = 1\dots r\,,\\
    g_{(q+r+l)}(x) &= {-}h_l(x) - \tau \leq 0,&&\quad l = 1\dots r\,.
  \end{aligned}
\end{equation*}
Thus, $r$ equality constraints become $2r$ inequality constraints, raising the total number of constraints to $s = q +
2r$. For each solution $p_i$, $c_i$ denotes the constraint vector where, $c_{ik} = \max\{g_k(p_i), 0\},~k=1\dots s$.
When $c_{ik} = 0,~\forall k=1\dots s$, the solution $p_i$ lies within the feasible region. When $c_{ik} > 0$, the
solution $p_i$ violates the $k$\tsup{th} constraint.

Under these guidelines, the representation of CDHS estimation (described in Section~\ref{subsec:cdhs}) under CRS as a
CO problem is given by,

\begin{equation}\label{eq:cdhscrs}
  \begin{aligned}
    & \underset{\alpha,\beta,\gamma,\delta}{\text{minimize:}}
    & Y_i = {-}R^\alpha.D^\beta,\ Y_s = {-}{V_e}^\gamma.{T_s}^\delta\\
    & \text{subject to:}
    &  {-}\phi + \epsilon \leq 0,\quad\phi - 1 + \epsilon \leq 0\,, &\quad\quad \forall \phi\in\{\alpha,\beta,\gamma,\delta\}\\
    && (\alpha+\beta-1) - \tau \leq 0,\quad(\gamma+\delta-1) - \tau \leq 0\,,\\
    && (1-\alpha-\beta) - \tau \leq 0,\quad(1-\gamma-\delta) - \tau \leq 0\,.
  \end{aligned}
\end{equation}
Under DRS the last two constraints for $Y_i$ and $Y_s$ are replaced with,
\begin{equation}\label{eq:cdhsdrs}
  \begin{aligned}
    \alpha + \beta + \epsilon - 1 &\leq 0\,,\\
    \gamma + \delta + \epsilon - 1 &\leq 0\,.
  \end{aligned}
\end{equation}

The representation of CEESA score estimation (described in Section~\ref{subsec:ceesa}) under DRS as a CO problem is
given by,
\begin{equation}\label{eq:ceesadrs}
  \begin{aligned}
    & \underset{r,d,t,v,e,\rho,\eta}{\text{minimize}}
    & Y = {-}{(r.R^\rho+d.D^\rho+t.{T_s}^\rho+v.{V_e}^\rho+e.E^\rho)}^{\frac{\eta}{\rho}}\\
    & \text{subject to}
    &   {-}\phi + \epsilon \leq 0,\quad\phi - 1 + \epsilon \leq 0\,, &\quad\quad \forall \phi\in\{r,d,t,v,e,\eta\}\\
    && \rho - 1 \leq 0,\quad \rho - 1 + \epsilon \leq 0\,,\\
    && (r+d+t+v+e-1) - \tau \leq 0\,,\\
    && (1-r-d-t-v-e) - \tau \leq 0\,.
  \end{aligned}
\end{equation}
Under CRS there is no need for the parameter $\eta$ (since $\eta=1$). Thus, the objective function for the problem reduces to,
\begin{equation}\label{eq:ceesacrs}
  \underset{r,d,t,v,e,\rho}{\text{minimize}}\quad
  Y = {-}{(r.R^\rho+d.D^\rho+t.{T_s}^\rho+v.{V_e}^\rho+e.E^\rho)}^{\frac{1}{\rho}}\,.
\end{equation}


\section{Experiment and Results}\label{sec:results}

\begin{table}
  \begin{center}
    \begin{tabular}{l l l}
      \toprule
      \textbf{Parameter} & \textbf{Description} & \textbf{Unit}\\
      \midrule
      P. Radius       & Estimated radius         & Earth Units (EU)\\
      P. Density      & Density                  & Earth Units (EU)\\
      P. Esc Vel      & Escape velocity          & Earth Units (EU)\\
      P. Ts Mean      & Mean Surface temperature & Kelvin (K)\\
      P. Eccentricity & Orbital eccentricity\\
      \bottomrule
    \end{tabular}
  \end{center}
  \caption{Parameters from the PHL-EC used for the experiment.}\label{tab:param}
\end{table}


\begin{figure}
  \centering
  \begin{subfigure}[b]{0.38\textwidth}
    \includegraphics[width=\textwidth]{dcdhscrs.png}
    \caption{CRS Score Distribution}\label{fig:distcdcrs}
  \end{subfigure}
  \quad
  \begin{subfigure}[b]{0.38\textwidth}
    \includegraphics[width=\textwidth]{dcdhsdrs.png}
    \caption{DRS Score Distribution}\label{fig:distcddrs}
  \end{subfigure}

  \begin{subfigure}[b]{0.38\textwidth}
    \includegraphics[width=\textwidth]{icdhscrs.png}
    \caption{CRS Iterations Distribution}\label{fig:itercdcrs}
  \end{subfigure}
  \quad
  \begin{subfigure}[b]{0.38\textwidth}
    \includegraphics[width=\textwidth]{icdhsdrs.png}
    \caption{DRS Iterations Distribution}\label{fig:itercddrs}
  \end{subfigure}
  \caption{Plots for the Cobb-Douglas Habitability Score.}\label{fig:cdhs}
\end{figure}


\begin{figure}
  \centering
  \begin{subfigure}[b]{0.38\textwidth}
    \includegraphics[width=\textwidth]{dceesacrs.png}
    \caption{CRS Score Distribution}\label{fig:distcecrs}
  \end{subfigure}
  \quad
  \begin{subfigure}[b]{0.38\textwidth}
    \includegraphics[width=\textwidth]{dceesadrs.png}
    \caption{DRS Score Distribution}\label{fig:distcedrs}
  \end{subfigure}

  \begin{subfigure}[b]{0.38\textwidth}
    \includegraphics[width=\textwidth]{iceesacrs.png}
    \caption{CRS Iterations Distribution}\label{fig:itercecrs}
  \end{subfigure}
  \quad
  \begin{subfigure}[b]{0.38\textwidth}
    \includegraphics[width=\textwidth]{iceesadrs.png}
    \caption{DRS Iterations Distribution}\label{fig:itercedrs}
  \end{subfigure}
  \caption{Plots for the Constant Elasticity Earth Similarity Approach.}\label{fig:ceesa}
\end{figure}


The data set used for estimating the Habitability Scores of exoplanets was the Confirmed Exoplanets Catalog maintained
by the Planetary Habitability Laboratory (PHL)~\cite{PHL}. The catalog records observed and modeled parameters for
exoplanets confirmed by the Extrasolar Planets Encyclopedia. Table~\ref{tab:param} describes the parameters from the PHL
Exoplanets' Catalog (PHL-EC) used for the experiment. Since surface temperature and eccentricity are not recorded in
Earth Units, we normalized these values by dividing them with Earth's surface temperature (\SI{288}{\kelvin}) and
eccentricity (\num{0.017}). PHL-EC assumes an Eccentricity of 0 when unavailable. The PHL-EC records empty values for
planets whose surface temperature is not known. We chose to drop these records from the experiment.

Our implementation used $n=25$ particles to traverse the search space, with learning rates $\lambda_g=0.8$ and
$\lambda_p=0.2$. It used an inertial weight of $\omega=0.6$ and upper and lower bounds $\pm1.0$. We used an error
threshold of $\epsilon=\num{1e-6}$ to convert strict inequalities to non-strict inequalities, and a tolerance of
$\tau=\num{1e-7}$ to transform an equality constraint to a pair of inequalities. Further implementation details are
discussed in the Appendix.

The plots in Figures~\ref{fig:distcdcrs} and~\ref{fig:distcddrs} describe the distribution of the CDHS across exoplanets
tested from the PHL-EC.\@ Figures~\ref{fig:itercdcrs} and~\ref{fig:itercddrs} show the distribution of iterations
required to converge to a global maxima. The spike at 0 is caused by particles converging to a $\gbest$ that does not
shift from the original position (for a more detailed explanation see Appendix~\ref{app:con}). The plots in
Figures~\ref{fig:distcedrs} and~\ref{fig:distcedrs} describe the distribution of the CEESA score across the exoplanets,
while Figures~\ref{fig:itercedrs} and~\ref{fig:itercecrs} show the distribution of iterations to convergence. These
graphs aggregate the results of optimizing the Habitability Production Functions
(Equations~\ref{eq:cdhscrs},~\ref{eq:cdhsdrs},~\ref{eq:ceesadrs} and~\ref{eq:ceesacrs}) for each exoplanet in the PHL-EC
by the method described in Algorithm~\ref{alg:cop}.

Tables~\ref{tab:cdhscrs} and~\ref{tab:cdhsdrs} record the CDHS values for a sample of exoplanets under CRS and DRS
respectively at $w_i=0.99$ and $w_s=0.01$. Tables~\ref{tab:ceesacrs} and~\ref{tab:ceesadrs} record the same for the
CEESA scores. All tables also record the number of iterations taken to converge to a stable $\gbest$ value. As seen in
the tables, although CEESA has 7 parameters and 16 constraints under DRS, PSO takes a little over twice the number of
iterations to converge as in each step of CDHS, which has 2 parameters and 5 constraints. This is a promising result as
it indicates that the iterations required for converging increases sub-linearly with the number of parameters in the
model.

As for real time taken to converge, PSO took \SI{666.85}{\second} ($\approx\SI{11}{\minute}\ \SI{7}{\second}$) to
estimate the CDHS under CRS for \num{1683} exoplanets, at an average of \SI{198.11}{\milli\second} for each planet for
each individual score (interior and surface) of the CDHS.\@ For CDHS under DRS, it took \SI{638.69}{\second}
($\approx\SI{10}{\minute}\ \SI{39}{\second}$) at an average of \SI{189.75}{\milli\second} for each part of the CDHS.\@
The CEESA calculations, requiring a single estimate, took a little over half the CDHS execution time to run. Under DRS
it took a total of \SI{370.86}{\second} ($\approx\SI{6}{\minute}\ \SI{11}{\second}$) at \SI{220.36}{\milli\second} per
planet, while under CRS it took \SI{356.92}{\second} ($\approx\SI{5}{\minute}\ \SI{57}{\second}$) at
\SI{212.07}{\milli\second} per planet.

\begin{table}
  \centering
  \begin{subtable}{\textwidth}
    \centering
    \begin{tabular}{l r r r r r r r r r r}
  \toprule
  Name & Class & $\alpha$ & $\beta$ & $Y_i$ & $i_i$ & $\gamma$ & $\delta$ & $Y_s$ & $i_s$ & $\mathit{CDHS}$\\
  \midrule
  GJ 176 b & non & $0.460$ & $0.540$ & $1.90$ & $50$ & $0.107$ & $0.893$ & $2.11$ & $61$ & $1.90$\\
  GJ 667 C b & non & $0.423$ & $0.577$ & $1.71$ & $58$ & $0.692$ & $0.308$ & $1.81$ & $54$ & $1.71$\\
  GJ 667 C e & psy & $0.129$ & $0.871$ & $1.40$ & $50$ & $0.258$ & $0.742$ & $1.39$ & $55$ & $1.40$\\
  GJ 667 C f & psy & $0.534$ & $0.466$ & $1.40$ & $48$ & $0.865$ & $0.135$ & $1.39$ & $47$ & $1.40$\\
  GJ 3634 b & non & $0.409$ & $0.591$ & $1.89$ & $58$ & $0.724$ & $0.276$ & $2.09$ & $48$ & $1.89$\\
  HD 20794 c & non & $0.260$ & $0.740$ & $1.35$ & $50$ & $0.096$ & $0.904$ & $1.34$ & $58$ & $1.35$\\
  HD 40307 e & non & $0.168$ & $0.832$ & $1.50$ & $49$ & $0.636$ & $0.364$ & $1.53$ & $63$ & $1.50$\\
  HD 40307 f & non & $0.702$ & $0.298$ & $1.52$ & $68$ & $0.303$ & $0.697$ & $1.55$ & $45$ & $1.52$\\
  HD 40307 g & psy & $0.964$ & $0.036$ & $1.82$ & $51$ & $0.083$ & $0.917$ & $1.98$ & $55$ & $1.82$\\
  Kepler-186 f & hyp & $0.338$ & $0.662$ & $1.17$ & $50$ & $0.979$ & $0.021$ & $1.12$ & $40$ & $1.17$\\
  Proxima Cen b & psy & $0.515$ & $0.484$ & $1.12$ & $37$ & $0.755$ & $0.245$ & $1.07$ & $ 0$ & $1.12$\\
  TRAPPIST-1 b & non & $0.319$ & $0.681$ & $1.09$ & $ 0$ & $0.801$ & $0.199$ & $0.89$ & $ 0$ & $1.09$\\
  TRAPPIST-1 c & non & $0.465$ & $0.535$ & $1.06$ & $ 0$ & $0.935$ & $0.065$ & $1.14$ & $26$ & $1.06$\\
  TRAPPIST-1 d & mes & $0.635$ & $0.365$ & $0.77$ & $34$ & $0.475$ & $0.525$ & $0.73$ & $47$ & $0.77$\\
  TRAPPIST-1 e & psy & $0.145$ & $0.855$ & $0.92$ & $ 0$ & $0.897$ & $0.103$ & $0.83$ & $55$ & $0.92$\\
  TRAPPIST-1 g & hyp & $0.226$ & $0.774$ & $1.13$ & $43$ & $0.876$ & $0.124$ & $1.09$ & $ 0$ & $1.13$\\
  \bottomrule\\
\end{tabular}

    \caption{Estimated habitability scores under CRS.}\label{tab:cdhscrs}
  \end{subtable}

  \vspace{4em}

  \begin{subtable}{\textwidth}
    \centering
    \begin{tabular}{l r r r r r r r r r r}
  \toprule
  Name & Class & $\alpha$ & $\beta$ & $Y_i$ & $i_i$ & $\gamma$ & $\delta$ & $Y_s$ & $i_s$ & $\mathit{CDHS}$\\
  \midrule
  GJ 176 b & non & $0.395$ & $0.604$ & $1.90$ & $59$ & $0.372$ & $0.627$ & $2.11$ & $56$ & $1.90$\\
  GJ 667 C b & non & $0.781$ & $0.218$ & $1.71$ & $58$ & $0.902$ & $0.097$ & $1.81$ & $57$ & $1.71$\\
  GJ 667 C e & psy & $0.179$ & $0.820$ & $1.40$ & $49$ & $0.234$ & $0.765$ & $1.39$ & $60$ & $1.40$\\
  GJ 667 C f & psy & $0.704$ & $0.295$ & $1.40$ & $64$ & $0.398$ & $0.601$ & $1.39$ & $61$ & $1.40$\\
  GJ 3634 b & non & $0.602$ & $0.397$ & $1.89$ & $59$ & $0.429$ & $0.570$ & $2.09$ & $77$ & $1.89$\\
  HD 20794 c & non & $0.014$ & $0.985$ & $1.35$ & $50$ & $0.116$ & $0.883$ & $1.34$ & $45$ & $1.35$\\
  HD 40307 e & non & $0.752$ & $0.247$ & $1.50$ & $60$ & $0.677$ & $0.322$ & $1.53$ & $50$ & $1.50$\\
  HD 40307 f & non & $0.887$ & $0.112$ & $1.52$ & $51$ & $0.261$ & $0.738$ & $1.55$ & $60$ & $1.52$\\
  HD 40307 g & psy & $0.300$ & $0.699$ & $1.82$ & $62$ & $0.785$ & $0.214$ & $1.98$ & $56$ & $1.82$\\
  Kepler-186 f & hyp & $0.073$ & $0.926$ & $1.17$ & $46$ & $0.740$ & $0.259$ & $1.12$ & $51$ & $1.17$\\
  Proxima Cen b & psy & $0.045$ & $0.954$ & $1.12$ & $57$ & $0.216$ & $0.783$ & $1.07$ & $53$ & $1.12$\\
  TRAPPIST-1 b & non & $0.102$ & $0.897$ & $1.09$ & $41$ & $0.000$ & $0.000$ & $1.00$ & $65$ & $1.09$\\
  TRAPPIST-1 c & non & $0.471$ & $0.528$ & $1.06$ & $44$ & $0.227$ & $0.772$ & $1.14$ & $57$ & $1.06$\\
  TRAPPIST-1 d & mes & $0.000$ & $0.000$ & $1.00$ & $67$ & $0.000$ & $0.000$ & $1.00$ & $59$ & $1.00$\\
  TRAPPIST-1 e & psy & $0.000$ & $0.000$ & $1.00$ & $55$ & $0.000$ & $0.000$ & $1.00$ & $57$ & $1.00$\\
  TRAPPIST-1 g & hyp & $0.888$ & $0.111$ & $1.13$ & $47$ & $0.949$ & $0.050$ & $1.09$ & $46$ & $1.13$\\
  \bottomrule\\
\end{tabular}

    \caption{Estimated habitability scores under DRS.}\label{tab:cdhsdrs}
  \end{subtable}
  \caption{Cobb-Douglas Habitability Scores as estimated by \pso.\\
    \footnotesize $\alpha$, $\beta$, $\gamma$ and $\delta$ record the parameters of Equation~\ref{eq:cdhscrs} in
    Table~\ref{tab:cdhscrs} and the parameters of Equation~\ref{eq:cdhsdrs} in Table~\ref{tab:cdhsdrs}. $Y_i$ and $Y_s$
    record the maxima for the objective functions. $i_i$ and $i_s$ specify the number of iterations taken to converge to
    a stable $\gbest$ value. Under the Class column there are four categories for the planets --- Psychroplanets (psy),
    Mesoplanets (mes), Non-Habitable planets (non) and Hypopsychroplanets (hyp). $\mathit{CDHS}$ is the final score
    with $w_i=0.99$ and $w_s=0.01$.
  }
\end{table}


\begin{table}
  \centering
  \begin{subtable}{\textwidth}
    \centering
    \begin{tabular}{l r r r r r r r r r r}
  \toprule
  Name & Class & $r$ & $d$ & $t$ & $v$ & $e$ & $\rho$ & $\eta$ & $\mathit{CDHS}$ & $i$\\
  \midrule
GJ 176 b & non & $0.304$ & $0.001$ & $0.375$ & $0.271$ & $0.050$ & $0.467$ & $0.808$ & $1.52$ & $ 85$\\
GJ 667 C b & non & $0.297$ & $0.010$ & $0.318$ & $0.052$ & $0.322$ & $0.682$ & $0.730$ & $2.36$ & $ 90$\\
GJ 667 C e & psy & $0.230$ & $0.286$ & $0.137$ & $0.199$ & $0.148$ & $0.551$ & $0.906$ & $1.14$ & $ 85$\\
GJ 667 C f & psy & $0.397$ & $0.035$ & $0.152$ & $0.402$ & $0.014$ & $0.793$ & $0.999$ & $1.31$ & $100$\\
GJ 3634 b & non & $0.178$ & $0.175$ & $0.005$ & $0.194$ & $0.447$ & $0.894$ & $0.657$ & $2.07$ & $ 94$\\
HD 20794 c & non & $0.073$ & $0.142$ & $0.452$ & $0.190$ & $0.144$ & $0.953$ & $0.635$ & $1.20$ & $ 78$\\
HD 40307 e & non & $0.156$ & $0.307$ & $0.185$ & $0.033$ & $0.319$ & $0.428$ & $0.939$ & $2.69$ & $ 88$\\
HD 40307 f & non & $0.272$ & $0.231$ & $0.064$ & $0.305$ & $0.127$ & $0.676$ & $0.802$ & $1.28$ & $ 77$\\
HD 40307 g & psy & $0.113$ & $0.219$ & $0.066$ & $0.454$ & $0.148$ & $0.711$ & $0.991$ & $3.26$ & $ 92$\\
Kepler-186 f & hyp & $0.039$ & $0.159$ & $0.116$ & $0.329$ & $0.357$ & $0.253$ & $0.919$ & $1.35$ & $ 70$\\
Proxima Cen b & psy & $0.272$ & $0.173$ & $0.284$ & $0.193$ & $0.079$ & $0.615$ & $0.114$ & $0.99$ & $ 75$\\
TRAPPIST-1 b & non & $0.488$ & $0.151$ & $0.039$ & $0.193$ & $0.129$ & $0.151$ & $0.014$ & $0.99$ & $ 87$\\
TRAPPIST-1 c & non & $0.172$ & $0.236$ & $0.275$ & $0.242$ & $0.075$ & $0.969$ & $0.962$ & $1.06$ & $ 80$\\
TRAPPIST-1 d & mes & $0.106$ & $0.308$ & $0.075$ & $0.218$ & $0.293$ & $0.844$ & $0.017$ & $0.99$ & $ 93$\\
TRAPPIST-1 e & psy & $0.189$ & $0.266$ & $0.192$ & $0.094$ & $0.260$ & $0.371$ & $0.006$ & $0.99$ & $ 84$\\
TRAPPIST-1 g & hyp & $0.326$ & $0.186$ & $0.143$ & $0.278$ & $0.067$ & $0.315$ & $0.021$ & $1.00$ & $ 76$\\
  \bottomrule\\
\end{tabular}



    \caption{Estimated habitability scores under DRS.}\label{tab:ceesadrs}
  \end{subtable}

  \vspace{4em}

  \begin{subtable}{\textwidth}
    \centering
    \begin{tabular}{l r r r r r r r r r r}
  \toprule
  Name & Class & $r$ & $d$ & $t$ & $v$ & $e$ & $\rho$ & $\eta$ & $\mathit{CDHS}$ & $i$\\
  \midrule
GJ 176 b & non & $0.194$ & $0.020$ & $0.315$ & $0.465$ & $0.006$ & $0.398$ & $1.000$ & $1.88$ & $ 86$\\
GJ 667 C b & non & $0.162$ & $0.289$ & $0.090$ & $0.087$ & $0.372$ & $0.836$ & $1.000$ & $3.54$ & $107$\\
GJ 667 C e & psy & $0.373$ & $0.032$ & $0.134$ & $0.304$ & $0.157$ & $0.217$ & $1.000$ & $1.25$ & $ 71$\\
GJ 667 C f & psy & $0.394$ & $0.006$ & $0.043$ & $0.360$ & $0.196$ & $0.490$ & $1.000$ & $1.44$ & $ 81$\\
GJ 3634 b & non & $0.351$ & $0.122$ & $0.006$ & $0.069$ & $0.453$ & $0.439$ & $1.000$ & $2.89$ & $ 96$\\
HD 20794 c & non & $0.101$ & $0.077$ & $0.691$ & $0.071$ & $0.059$ & $0.756$ & $1.000$ & $1.58$ & $ 94$\\
HD 40307 e & non & $0.069$ & $0.091$ & $0.097$ & $0.173$ & $0.569$ & $0.768$ & $1.000$ & $5.29$ & $ 94$\\
HD 40307 f & non & $0.285$ & $0.161$ & $0.053$ & $0.443$ & $0.058$ & $0.342$ & $1.000$ & $1.42$ & $ 73$\\
HD 40307 g & psy & $0.156$ & $0.010$ & $0.081$ & $0.302$ & $0.451$ & $0.612$ & $1.000$ & $7.15$ & $ 94$\\
Kepler-186 f & hyp & $0.036$ & $0.017$ & $0.082$ & $0.383$ & $0.483$ & $0.929$ & $1.000$ & $1.68$ & $ 85$\\
Proxima Cen b & psy & $0.352$ & $0.383$ & $0.103$ & $0.059$ & $0.103$ & $0.936$ & $1.000$ & $0.89$ & $ 83$\\
TRAPPIST-1 b & non & $0.148$ & $0.147$ & $0.344$ & $0.269$ & $0.093$ & $0.767$ & $1.000$ & $0.94$ & $ 81$\\
TRAPPIST-1 c & non & $0.038$ & $0.060$ & $0.575$ & $0.321$ & $0.005$ & $0.602$ & $1.000$ & $1.17$ & $ 86$\\
TRAPPIST-1 d & mes & $0.023$ & $0.065$ & $0.475$ & $0.391$ & $0.045$ & $0.830$ & $1.000$ & $0.84$ & $ 79$\\
TRAPPIST-1 e & psy & $0.176$ & $0.464$ & $0.253$ & $0.103$ & $0.004$ & $0.920$ & $1.000$ & $0.86$ & $ 81$\\
TRAPPIST-1 g & hyp & $0.060$ & $0.086$ & $0.310$ & $0.540$ & $0.004$ & $0.848$ & $1.000$ & $0.97$ & $ 86$\\
  \bottomrule\\
\end{tabular}



    \caption{Estimated habitability scores under CRS.}\label{tab:ceesacrs}
  \end{subtable}
  \caption{Constant Elasticity Earth Similarity Approach scores as estimated by \pso.\\
      \footnotesize $r$, $d$, $t$, $v$, $e$, $\rho$ and $eta$ record the parameters of Equation~\ref{eq:ceesadrs} in
      Table~\ref{tab:ceesadrs} and the parameters of Equation~\ref{eq:ceesacrs} in Table~\ref{tab:ceesacrs}. However,
      since under the CRS constraint, $\eta=1$, there is no need for the parameter $\eta$ in the problem. The column
      $\mathit{CEESA}$ records the maxima of the objective function. $i$ specifies the number of iterations taken to
      converge to the maxima.
    }
\end{table}


\section{Conclusion}\label{sec:conc}

\pso\ mainly draws its advantages from being easy to implement and highly parallelizable. The algorithms described in
Section~\ref{sec:pso} use simple operators and straightforward logic. What is especially noticeable is the lack of the
need for a gradient, allowing PSO to work in high dimensional search spaces with a large number of constraints,
precisely what is needed in a potential Habitability score estimate. Further, particles of the swarm, in most
implementations operate independently during each iteration, their updates can occur simultaneously and even
asynchronously, yielding much faster execution times than those outlined in Section~\ref{sec:results}. However, since
strict inequalities and equality constraints are not exactly represented, the resulting solution may not be as accurate
as direct methods. Despite this, using PSO to calculate the habitability scores is beneficial when the number of input
parameters are large, which further increases the number of constraints, resulting in a model too infeasible for
traditional optimization methods.


\printbibliography


\begin{appendices}

  \section{Ensuring Convergence}\label{app:con}

  Although the termination criteria illustrated in Algorithm~\ref{alg:unop}, $|\oldbest-\gbest| < \mathit{threshold}$,
  is a lucrative solution for unconstrained optimization, it might cause early termination for a constrained problem.
  This could occur in Algorithm~\ref{alg:cop} as the particles traverse the search space. Due to the way they are
  updated, all individual particles could move to infeasible regions. There are no updates made to their corresponding
  $\pbest$s, leading to no change in $\gbest$. Thus, although the swarm has not yet converged to the global optima, it
  terminates and reports the current $\gbest$ as the optima.

  To prevent this from happening our implementation maintained a counter that it incremented every time $\gbest$ changed
  by a value less than the threshold. When the counter reached \num{100}, the algorithm terminated and reported the
  current value of $\gbest$ as the global optima. If the value of $\gbest$ changed by more than the threshold, it reset
  the counter to $\num{0}$. Since the function is convex, and all points are initialized to feasible solutions (thus,
  ensuring that a feasible $\gbest$ has been encountered) there was no need to restrict the total number of possible
  iterations.

  This explains the spike at \num{0} in Figure~\ref{fig:itercdcrs}. For every exoplanet, the implementation waits a
  \num{100} iterations for declaring convergence, thus overstating the total iterations by \num{100}. To adjust for this
  offset, we subtracted \num{100} from the total iterations before constructing the histogram plots in
  Figures~\ref{fig:itercdcrs},~\ref{fig:itercddrs},~\ref{fig:itercecrs} and~\ref{fig:itercedrs}. Therefore, although the
  graphs report iterations to convergence for a significant number of planets, only $\gbest$ has not shifted, the
  particles have actually converged to a common point around a region within the threshold of the global best.

  \section{Improving Performance}\label{app:imp}

  Most programming languages today have either built-in or external library support for linear algebra routines. The
  implementation of these libraries are well-tested and efficient, and take advantage of system features such as
  multiple cores or a general purpose GPU computing device. Taking advantage of these libraries requires representing
  most entities as either matrices or vectors. 

  Matrices $P$ and $V$ represent current position and velocity of all particles in the swarm.
  \begin{alignat*}{2}
    P &= [\ p_{ij}\ &&\mid\ \forall i=1\dots n,\;\forall j=1\dots d\ ]\,,\\
    V &= [\ v_{ij}\ &&\mid\ \forall i=1\dots n,\;\forall j=1\dots d\ ]\,.
  \end{alignat*}
  $p_{ij}$ and $v_{ij}$ denote position and velocity of particle $i$ in dimension $d$. Matrices $B$ and $L$ represent
  the $\pbest$ and $\lbest$ values for the swarm, where,
  \begin{align*}
    B &= [\ {\pbest}_i\ \mid\ \forall i=1\dots n\ ]\,,\\
    L &= [\ \argmin_{B_j,\ j = 1\dots n} {\|B_j - P_i\|}^2\ \mid\ \forall i=1\dots n\ ]\,.
  \end{align*}
  Section~\ref{sec:rep} outlined the use of a constraint vector for describing the feasibility of a solution. This
  concept generalizes to a matrix $C$, where,
  \begin{equation*}
    C = [\ c_{ij}\ \mid\ \forall i=1\dots n,\;\forall j=1\dots s\ ]\,.
  \end{equation*}
  Let $r',r''$ be two random vectors of length $n$ drawn from the uniform distribution ${U(0,1)}^n$. We use the
  shorthand notation of $X_i$ to denote the $i$\tsup{th} row of some matrix $X$. If $f(X_i)$ is the fitness function and
  $\mathit{constraints}(X)$ constructs $C$ for solutions $X$, the update of each particle in Algorithm~\ref{alg:cop} is
  implemented as,
  \begin{align*}
    V  =&\ \omega.V + \lambda_g\begin{bmatrix}
      {r_1}'(\gbest - P_1)\\
      {r_2}'(\gbest - P_2)\\
      \vdots\\
      {r_n}'(\gbest - P_n)
    \end{bmatrix} + \lambda_p\begin{bmatrix}
      {r_1}''(L_1 - P_1)\\
      {r_2}''(L_2 - P_2)\\
      \vdots\\
      {r_n}''(L_n - P_n)
    \end{bmatrix}\\
    V  =&\ \left[\ \sgn(v_{ij})*\max\{|v_{max}|,|v_{ij}|\}\ \mid\ \forall i=1\dots n,\;\forall j=1\dots d\ \right]\\
    P  =&\ P + V\\
    C  =&\ \mathit{constraints}(P)\\
    B  =&\ [\ X_i\ \mid\ (\textstyle\sum_{j=1}^s c_{ij} = 0 \to X_i = \argmax\{f(P_i), f(B_i)\})\ \wedge\\
        &\qquad(\textstyle\sum_{j=1}^s c_{ij} \neq 0 \to X_i = B_i),\ \forall i=1\dots n\ ]\\
  \end{align*}
  Finally, after all particles are updated, $\gbest$ is updated according to,
  \begin{equation*}
    \gbest = \argmax_{{B}_i} \{\ {f(B_i)}\ \mid\ \forall i=\dots n\ \}\,.
  \end{equation*}

\end{appendices}

\end{document}
